\setstretch{1.3}  % ← 2차 선언 (본문에 적용 보장)
\section{Introduction}

양자컴퓨팅은 지난 수년 동안 급속한 발전을 이루어 왔다. 1980년대의 개념적 구상 \cite{Benioff1980, Feynman1982}에서부터, 2000년대 하드웨어에 대한 초기 원리 증명 \cite{Chuang1998, Jones1999, Leung2000, Vandersypen2001, Haffner2005, Negrevergne2006, Plantenberg2007, Hanneke2009, Monz2011, Devitt2013, Devitt2016, Monz2016}을 거쳐, 이제는 수백 개의 qubit으로 구성된 양자컴퓨터를 구축할 수 있게 되었다 \cite{Jurcevic2021, Pino2021, Ebadi2021}. 이 기술은 아직 초기 단계에 머무르고 있으나, 양자 하드웨어의 빠른 발전과 전 세계적인 대규모 재정 투자로 인해, 이른바 Noisy-Intermediate Scale Quantum (NISQ) 장치들이 머지않아 기존의 컴퓨터를 능가할 수 있을 것이라는 주장이 제기되고 있다 \cite{preskillQuantumComputingNISQ2018, Brooks2019, Arute2019, Zhong2020, Wu2021, Madsen2022}. NISQ 장치는 단기적인 양자컴퓨터로, qubit 수가 제한적이며 견고한 오류 정정 기법을 구현하기에는 물리적 qubit 수가 부족하다. 현재의 NISQ 컴퓨터들은 이미 제한된 문제 집합에 대해 기존 컴퓨터보다 우수한 성능을 보인 바 있으며, 이 문제들은 양자컴퓨터의 특성에 맞추어 특별히 설계된 것들이다 \cite{Arute2019, Zhong2020, Wu2021}. 이러한 제약된 장치에서 구동되는 알고리즘들은 소수의 qubit만을 요구하며, 일정 수준의 노이즈 내성을 갖추고 있고, 일부 계산 단계를 양자 장치에서 수행하고 나머지를 기존 컴퓨터에서 수행하는 하이브리드 알고리즘으로 구성되는 경우가 많다. 특히, 연산 수, 즉 quantum gate의 수는 중간 수준으로 유지되어야 하는데, 이는 연산 구현에 시간이 오래 걸릴수록 양자 상태에 더 많은 오류가 도입되고 탈동조화될 가능성이 커지기 때문이다. 이러한 제약으로 인해 고려 가능한 알고리즘의 범위는 크게 제한된다. 예를 들어 소인수분해를 위한 Shor의 알고리즘 \cite{shorAlgorithmsQuantumComputation, Vandersypen2001, ChaoYang2007, Lanyon2007, Lucero2012, MartinLopez2012, Markov2013, Amico2019}이나 비구조적 탐색 문제를 위한 Grover의 알고리즘 \cite{groverFastQuantumMechanical1996, Bennett1997, Cerf2000, Ambainis2004, Ambainis2007, Bernstein2010}은 적합하지 않다.

Variational Quantum Eigensolver (VQE)는 Peruzzo {\it et al.} \cite{Peruzzo2014}에 의해 처음 개발되었으며, 이론적 틀은 McClean \textit{et al.}에 의해 확장되고 형식화되었다 \cite{mccleanTheoryVariationalHybrid2015}. VQE는 NISQ 알고리즘 중 가장 유망한 예시 중 하나이다. 가장 일반적인 설명에서, VQE는 Hamiltonian의 바닥상태 에너지에 대한 상계(upper bound)를 계산하는 것을 목표로 하며, 이는 일반적으로 분자 및 물질의 에너지 특성을 계산하는 첫 단계이다. 전자 구조 연구는 양자화학(예: \cite{Deglmann2014, WilliamsNoonan2017, Heifetz2020}) 및 응집물질물리학(예: \cite{Continentino2021, VanderVen2020})에서 핵심적인 응용 분야이다. 따라서 VQE의 적용 범위는 약물 발견 \cite{Cao2018_DD, Blunt2022}, 재료 과학 \cite{Lordi2021}, 화학 공학 \cite{Cao2019_QC} 등 매우 광범위하다. 전통적인 계산 화학은 거의 한 세기에 걸친 연구를 바탕으로 이러한 특성을 근사하는 효율적인 방법을 제공하지만, 매우 정확한 계산이 필요한 큰 계(system)에서는 계산 비용이 기하급수적으로 증가하여 실용적이지 않다. 이는 이러한 방법의 실제 응용에 도전을 야기한다. 계산 화학 기법이 분자계에서 충분한 정확도를 확보하지 못하는 주된 이유 중 하나는 구성 전자 간의 상관관계를 적절히 다루지 못하는 데 있다. 이러한 전자 간 상호작용은 형식적으로 계 크기에 대해 지수적으로 스케일하는 계산을 요구하며, 이는 기존 컴퓨팅으로는 정확한 양자화학 방법을 일반적으로 다루기 어렵게 만든다. 이러한 제한은 기존 컴퓨터에서 양자 시스템을 시뮬레이션하려는 문헌에서 잘 다루어졌으며, Ref.~\cite{Zhou2020}는 이에 대한 탁월한 사례를 제시한다.

이러한 병목 현상은 VQE와 같은 방법을 탐구하게 된 동기이며, 이러한 방법이 언젠가는 기존 컴퓨팅 패러다임을 능가할 수 있으리라는 기대와 연결되어 있다 \cite{Boixo2018, mccaskeyQuantumChemistryBenchmark2019}. 1982년, Richard Feynman은 양자 시스템의 시뮬레이션은 다른 양자 시스템을 제어하고 조작함으로써 가장 효율적으로 수행될 수 있다고 이론화하였다 \cite{Feynman1982}. qubit 배열은 전자파동함수와 마찬가지로 양자역학의 법칙을 따른다. 양자역학의 중첩 원리 \cite{Silverman2008, Ballentine2008}는 동일한 정보를 기존 장치에 부호화하려면 지수적으로 많은 자원이 필요하지만, qubit의 수는 선형적으로 증가하면 충분하다는 것을 의미한다. 전자 구조 이론의 맥락에서 \cite{Helgaker2000, Kratzer2019, Li2020_EST}, 이는 양자컴퓨팅의 매력으로 작용하며, 기존 컴퓨팅으로는 불가능한 양자 wavefunction을 정확하게 모델링하고 조작할 수 있는 가능성을 제공한다. VQE는 주로 전자 구조 연구에서 사용되고 있으나, 유사한 스케일링 문제를 가진 여러 양자역학적 문제에도 적용되어 왔다. 여기에는 원자핵물리학 \cite{Miceli2019, DiMatteo2021} 및 핵 구조 문제 \cite{Kiss2022, Romero2022}, 고에너지 물리학 \cite{Bauls2020, Bass2021, Bauer2022}, 진동 및 vibronic 분광학 \cite{McArdle2019_vibra, Sawaya2019, Ollitrault2020_vibrational, Jahangiri2020, Ltstedt2021, Sawaya2021}, 광화학 반응 특성 예측 \cite{Mitarai2020, Omiya2022}, 주기적 시스템 \cite{Liu2020, Yoshioka2022, Manrique2020}, 비선형 Schr{"{o}}dinger 방정식의 해석 \cite{Lubasch2020}, Schwarzschild-de Sitter 블랙홀 또는 Kantowski-Sachs 우주론의 양자 상태 계산 \cite{Joseph2022} 등이 포함된다.

VQE는 초기화된 qubit register로부터 시작한다. 그런 다음, 전자 wavefunction의 물리적 성질과 얽힘(entanglement)을 모델링하기 위해 이 register에 quantum circuit을 적용한다. quantum circuit은 qubit에 적용되는 일련의 사전 정의된 quantum operation을 의미한다 \cite{nielsenQuantumComputationQuantum2010}. circuit 내 연속 연산의 수는 깊이(depth)라고 한다. 이 회로는 (1) 정해진 순서의 quantum gate 집합으로 구성된 구조(ansatz), (2) 일부 gate의 동작을 결정짓는 매개변수 집합으로 정의된다. quantum circuit이 register에 적용된 후, qubit의 상태는 시도(trial) wavefunction을 모델링하도록 설계된다. 계의 Hamiltonian은 이 wavefunction에 대해 측정되어 에너지를 추정할 수 있다. VQE는 이 trial 에너지를 최소화하기 위해 ansatz의 매개변수를 변분적으로 최적화함으로써 작동하며, 이는 변분 원리 \cite{Rayleigh1870, Ritz1908, Arfken1985}에 따라 항상 정확한 바닥상태 에너지보다 높도록 제약된다. VQE가 실행 가능하려면 wavefunction을 모델링하는 데 필요한 quantum 연산 수가 충분히 낮아야 하므로 비교적 압축된 ansatz가 요구된다. VQE는 기존 컴퓨터로는 효율적으로 시뮬레이션할 수 없는 wavefunction ansatz를 허용하므로, 해당 quantum circuit이 충분히 정확한 trial wavefunction일 경우 기존 접근법에 비해 우위를 제공할 수 있다 \cite{Peruzzo2014}. 이러한 quantum ansatz의 가능성에 대한 첫 시연은 \cite{Peruzzo2014}에서 제시되었으며, 여기서는 qubit register 크기에 대해 다항식적으로 깊이가 증가하는 ansatz가 기존 양자화학의 원리에 기반하여 구성되었다(예: Unitary Coupled Cluster, 자세한 내용은 Sec. \ref{sec:UCCA}에서 논의됨). 이후로, qubit register 크기에 대해 선형적으로 스케일링되는 \cite{Lee2019} 등 다양한 ansatz가 제안되었다. 그러나 일반적으로 필요한 quantum 연산 수가 적은 얕은 ansatz는 가능한 wavefunction 공간의 범위를 더 적게 포괄하게 되므로, 바닥상태 에너지의 정확도가 낮을 수 있음을 이해해야 한다.

ansatz 설계는 VQE의 핵심이며, 이 접근 방식이 기존 방법에 비해 가지는 잠재적 이점의 이론적 기반과 직결된다. 그러나 이 외에도 알고리즘의 비용과 실행 가능성에 직접적인 영향을 미치는 다양한 구성 요소들이 존재한다. 본 리뷰에서는 이들 각각에 대한 논의를 제공하며, 여기서는 간략히 그 개요를 제시한다. 먼저 연구하고자 하는 양자 시스템을 선택한 후, 해당 시스템의 Hamiltonian을 어떻게 구성할지를 결정해야 한다. VQE에서는 Hamiltonian이 개별 항들의 합으로 표현될 수 있어야 하며, 이러한 항의 수는 시스템 크기에 대해 상대적으로 느리게 증가해야 한다. 이는 쿨롱 상호작용이 이체(two-body) 항들의 합으로 구성되어 다항식적으로 스케일링되는 전자 Hamiltonian의 일반적 특성이다 \cite{Szabo1996}. Hamiltonian의 수학적 표현 방식은 다양한 유연성과 선택지를 가지며, 이는 VQE의 모든 측면—qubit 수, ansatz의 깊이, 측정 횟수—에 영향을 미치기 때문에 매우 중요한 단계이다. Hamiltonian이 구성되면, 이를 quantum 컴퓨터에서 직접 측정 가능한 연산자(예: spin 또는 Pauli 연산자)로 변환해야 하며, 이 역시 ansatz의 깊이와 측정 횟수에 영향을 미칠 수 있다.

다음 단계는 ansatz를 선택하는 것이다. 이는 바닥상태 wavefunction을 근사적으로 모델링할 수 있을 정도로 충분히 표현력이 있어야 하지만, 회로가 너무 깊어지거나 학습해야 할 매개변수가 너무 많아지는 복잡성은 피해야 한다. ansatz가 정해지면, 적절한 최적화 알고리즘을 선택해야 하는데, 이는 VQE 최적화의 수렴 속도와 알고리즘 전체 비용에 중대한 영향을 미친다. quantum 장치에서의 측정은 본질적으로 확률적이므로 \cite{nielsenQuantumComputationQuantum2010}, 연산자의 기댓값을 얻기 위해 충분한 횟수의 반복 측정이 필요하다. 필요한 측정 횟수는 요구 정밀도뿐만 아니라 Hamiltonian의 연산자 수에도 영향을 받는다. 효율적인 측정 전략의 채택은 VQE 구현 비용을 제어하는 데 핵심적이다. 마지막으로, quantum 노이즈가 결과의 정확도에 미치는 영향을 줄이기 위해, 개별 측정에 오류 완화(error mitigation) 전략을 도입할 수 있으며, 이들 전략 역시 관련 계산 비용을 고려해야 한다.

본 리뷰의 목적은 두 가지이다: (1) 위에서 개요로 제시한 VQE 알고리즘의 다양한 구성 요소 각각에 대해 개관을 제공하고, 문헌에서 제안된 best practice를 제시하는 것이다. VQE는 매우 다양한 양자 시스템에 적용될 수 있기 때문에, 우리는 다음 두 가지 일반적 시스템 유형에 대해 best practice를 제안한다: \textit{ab initio} 분자 시스템과 spin-lattice 모델. 이 결론들은 불순물 모델, 응집물질, 진동 분광학 또는 핵 구조 등 다른 Hamiltonian에도 일반화될 수 있을 것으로 보인다. (2) VQE의 향후 적용 가능성과 앞서 소개한 알고리즘 구성 요소 각각에 대해 미해결 연구 질문들을 제시하는 것이다. 우리는 이 방법의 장래 활용 가능성에 영향을 미치는 네 가지 주요(그리고 상호 연관된) 난제를 식별하였으며, 이는 여기서 개략적으로 소개하고 본 리뷰 전반에 걸쳐 상세히 논의된다.

첫째, 여러 연구에서는 시스템 크기 증가에 따른 측정 횟수의 스케일링을 분석하였다 \cite{Wecker2015, Elfving2020, Kuhn2019, Gonthier2020}. 측정 횟수는 이론적으로는 다항식적으로 증가하지만 \cite{mccleanTheoryVariationalHybrid2015}, 실제로는 방법의 실행 가능성을 위협할 정도로 빠르게 커질 수 있다 \cite{Wecker2015, Elfving2020, Gonthier2020}. 이에 따라, 많은 연구에서 효율적인 측정 방식 개발의 필요성을 제기하였다 (예: Ref.\cite{Gonthier2020}), 특히 Hamiltonian의 간결한 표현을 찾는 접근이나, 가환하는 관측가능량을 동시에 측정하는 방식은 지속적으로 연구되고 있다 (자세한 내용은 Sec.\ref{sec:DecomposedInteractions} 참조).

둘째, 이러한 측정 문제에 대한 대안은 VQE의 병렬화 가능성에서 찾을 수 있다. 이 가능성은 Peruzzo {\it et al.}의 원 논문 \cite{Peruzzo2014}에서 명확히 제시되었으나, 병렬화 전략과 통신 오버헤드 측면에서 효율적인 방법론은 아직 커뮤니티에서 충분히 다루어지지 않았다 (관련 논의는 Sec.~\ref{sec:parallelization}에 있음).

셋째, variational quantum 알고리즘은 매개변수 최적화에서 공통적으로 등장하는 문제인 barren plateau 문제 \cite{Wecker2015, McClean2018, Cerezo2021_BP}에 직면한다. 이 문제는 VQE 매개변수의 그래디언트가 qubit 수, ansatz의 표현력 \cite{Holmes2016}, wavefunction의 얽힘 수준 \cite{OrtizMarrero2020, Patti2021}, 또는 cost function의 비국소성 \cite{Cerezo2021_BP, Uvarov2020, Sharma2020}에 따라 지수적으로 작아지는 현상으로 인해 발생하며, 대규모 시스템에서는 최적화를 불가능하게 만들 수 있다. 이를 해결하기 위한 다양한 방법이 제안되었지만, 이러한 기법들이 VQE 맥락에서 barren plateau 문제를 완전히 억제할 수 있는지는 여전히 불확실하다 (관련 논의는 Sec.~\ref{sec:barren_plateau} 참조). 이와 관련된 또 다른 문제는 VQE의 최적화 지형(landscape)의 복잡성이 아직 충분히 이해되지 않았다는 것이다. Bittel과 Kliesch \cite{Bittel2021}는 이 지형의 특정 특징을 규명하였으며, 지역 최소점 및 비볼록적(non-convex) 구조로 인해 최적화가 실패할 위험이 있음을 보였다. 궁극적인 질문은, 시스템 크기와 wavefunction의 얽힘 수준에 따라 수렴 속도와 최적화 반복 횟수가 어떻게 스케일링되는지에 대한 것이다.

마지막으로, VQE가 quantum 노이즈에 얼마나 강인한지, 그리고 이를 tractable한 방식으로 완화할 수 있는지에 대한 문제는 여전히 명확하지 않다. variational 알고리즘은 일부 체계적 노이즈를 학습적으로 제거할 수 있는 능력을 어느 정도 가지고 있음이 밝혀졌으나 \cite{Enrico2021EvaluatingNoiseResilience, Enrico2020NoiseInducedBreakingOfSymmetries}, 오류 완화 기법의 비용이 이점보다 큰 경우도 있다는 연구 결과도 존재한다 \cite{RyujiFundamentalLimits2021}. 따라서 VQE 결과에 대한 quantum 노이즈의 영향과 그 예측 가능성은 기존보다 훨씬 더 깊이 있는 연구가 필요하다. 전반적으로, 여러 장애 요소에도 불구하고 VQE는 NISQ 장치에서 최초로 실용화될 수 있는 방법 중 하나가 될 가능성이 있다. 그러나 하드웨어가 발전함에 따라, 이 접근법의 이론적 기반과 알고리즘 및 소프트웨어의 효율성 및 견고성도 함께 진보되어야만 VQE와 같은 기법의 잠재적 이점을 가능한 한 빠르게 활용할 수 있을 것이다.

\paragraph{Structure of the review:} Sec. \ref{sec:overview}에서는 VQE에 대한 보다 형식적인 정의와 그 잠재적 장점 및 한계를 논의하며, VQE의 최신 상태에 대한 평가를 제시한다. 이후의 절에서는 VQE 파이프라인의 각 구성 요소에 대한 리뷰를 제공한다 (도식은 Fig. \ref{fig:VQE_pipeline} 참조). 첫 단계는 분자 Hamiltonian의 표현 방식을 선택하는 것이며 (Sec. \ref{sec:Hamiltonian_representation}), 그 다음으로는 fermionic 연산자를 spin 연산자로 사상(mapping)하는 방법을 설명한다. 이는 Hamiltonian 연산자를 quantum 장치에서 직접 측정할 수 있게 만든다 (Sec. \ref{sec:Encoding}). 다음 절에서는 Hamiltonian의 기댓값을 추정하는 데 필요한 측정 횟수를 줄이기 위한 효율성 기법을 다룬다 (Sec. \ref{sec:Grouping}). 이후에는 다양한 wavefunction ans"atze에 해당하는 quantum circuit 구조들을 다룬다 (Sec. \ref{sec:Ansatz}). 이어서, VQE에 적용되는 주요 최적화 방법과 그 관련성을 설명한다 (Sec. \ref{sec:Optimization}). 마지막으로, 알고리즘의 전체 정확도를 향상시킬 수 있는 주요 오류 완화 기법을 소개한다 (Sec. \ref{sec:error-mit}). 리뷰의 마지막에서는 양자화학 응용을 위한 VQE의 확장 기법들을 논의한다 (Sec. \ref{sec:Extensions_of_VQE}).

\paragraph{Additional resources and other reviews:} Nielsen과 Chuang \cite{nielsenQuantumComputationQuantum2010}은 양자컴퓨팅과 양자정보의 기초를 다룬 입문서를 제공하며, 이는 여전히 이 분야의 고전적인 참고서로 평가받는다. 양자컴퓨팅의 개요와 현재 기술 수준에 대한 실용적 논의를 위해서는 Whitfield {\it et al.} \cite{Whitfield2022}의 리뷰를 추천한다. 더 간결한 입문 내용을 원한다면, McArdle {\it et al.} \cite{mcardleQuantumComputationalChemistry2018}의 리뷰를 참고할 수 있다. 이 문헌은 양자화학 및 재료과학에 대한 quantum computing 방법론을 포괄적으로 다룬다. 유사한 주제를 다룬 두 개의 리뷰는 Bauer {\it et al.} \cite{bauerQuantumAlgorithmsQuantum2020}와 Motta 및 Rice의 리뷰 \cite{Motta2021_review}이다. 보다 일반적인 variational quantum 알고리즘에 대한 리뷰로는 Cerezo {\it et al.} \cite{cerezoVariationalQuantumAlgorithms2020}의 논문을, 보다 넓은 NISQ 알고리즘 분야에 대해서는 Bharti {\it et al.} \cite{Bharti2021_review}의 리뷰를 추천한다. VQE와 관련된 구체적인 주제를 다룬 리뷰도 존재하며, 특히 ans{"{a}}tze에 대해 다룬 Fedorov {\it et al.} \cite{Fedorov2021}와 unitary coupled cluster 기반 ans{"{a}}tze를 다룬 Anand {\it et al.} \cite{Anand2021_review}의 리뷰는 본 논문의 분석을 보완한다. 마지막으로, 양자화학의 기초 개념에 대한 개관으로는 Szabo와 Ostlund의 저서 \cite{Szabo1996}를 추천한다.

VQE 분야는 매우 빠르게 확장되고 있으며, 본 리뷰는 2022년 5월 말까지의 문헌을 대상으로 한다.