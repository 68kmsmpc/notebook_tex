\section{Beyond the ground state of isolated molecules: Extensions of VQE} \label{sec:Extensions_of_VQE}

Up until this point, we have provided details of the VQE in the context of finding the ground state of an isolated system. However, in reality, a molecule is generally coupled with a wider environment, as well as the physics being strongly influenced by its electronic excitations. In this section, we briefly review some of the modifications to VQE which enlarge the scope of applicability, including accessing beyond ground state properties of the system, as well as the use of VQE as a sub-component of other algorithms to access multi-resolution descriptions of larger systems.

\subsection{Excited states VQE} \label{sec:excited_states}

The computation of excited states is key to many processes in quantum chemistry and materials science, governing the dominant optical, transport and reactive properties \cite{Karim2018, ZeinalipourYazdi2018}. However, it is in general a significantly more challenging task than ground state computation, owing to the state generally being further away from a mean-field description, as well as less straightforward optimization to avoid the variational collapse to the ground state. Conventional correlated quantum chemical approaches \cite{Matsika2018} include Equation of Motion (EOM) coupled-cluster \cite{Stanton1993}, linear response theory \cite{Monkhorst2009}, as well as multi-reference approaches for stronger correlation \cite{Jeziorski1981, Lyakh2011}. Quantum computing methods can be broadly divided into two main types of methods, those that rely on computing excited states within a subspace, and fully variational methods relying on modification of the VQE cost function. We briefly review the core aspects of some of these approaches below.

\paragraph{Quantum Subspace Expansion:} The quantum subspace expansion relies on finding an approximate Hamiltonian that spans a subspace of the full Hilbert space, but whose dimension is small and grows as only a low-order polynomial of the system size. The matrix elements of these Hamiltonians are sampled on quantum computers, but can then be tractably diagonalized on classical resources, with the higher-lying eigenvalues of these subspace Hamiltonians approximating true eigenvalues of the system. In practice, this approach starts with a ground state VQE calculation. From this ground state, it is then necessary to add additional states in order to define the span of a subspace into which the Hamiltonian can be computed. For reliable excited states, it is necessary to ensure that this space spans the dominant low-energy excitations of interest, as the whole spectrum will not be reproduced by construction. There are different approaches to choose these low energy states to span these relevant excitations, including approaches based on Krylov (or Lanczos) subspaces \cite{Motta2019,yeter-aydenizPracticalQuantumComputation2020,sunQuantumComputationFiniteTemperature2021}, and low-rank excitations of the ground state motivated by an equation-of-motion formalism \cite{McClean2017,Ollitrault2020}. These approaches can also be used to yield improved ground state estimates \cite{Motta2019,Parrish2019}.

In the quantum subspace expansion based around the equation-of-motion expansion, the resulting Hamiltonian (and overlap) matrix between these states can be found via high-order reduced density matrices evaluated from the ground state, as initially proposed in Ref.~\cite{McClean2017}, and subsequently implemented on a quantum device \cite{collessComputationMolecularSpectra2018}. The advantage of these methods is that they do not require particularly deep circuits to evaluate the relevant matrix elements of this subspace Hamiltonian. However, the quantum subspace expansion approaches can be quite sensitive to noise, while high-order density matrices can be expensive to sample and accumulate. Furthermore, noisy (yet unbiased) matrix elements can lead to systematic biases in eigenvalues \cite{Blunt2018, Epperly2021}.

\paragraph{Variational approaches:} An alternative approach relies on directly optimizing an ansatz for specific excited states, using a modified cost function, which affords a fully variational flexibility, while maintaining orthogonality to lower-energy states. These have the advantage of not suffering from the limitations and biases of subspace expansion methods, but usually come at a higher cost in terms of quantum resources, and a restriction to a specific ansatz chosen. The simplest approach is to simply enforce symmetry constraints on the ansatz to a different symmetry sector to the ground state, in which case orthogonality to the ground state is guaranteed for the lowest-lying excitations in each symmetry \cite{Ryabinkin2019} (for details about this method, please refer back to Sec. \ref{sec:cost_function}). This is however restricted to only specific excited states and limited by the symmetry of the system studied. Another approach which was proposed early on in the development of variational quantum algorithms (initially suggested for quantum computation in Ref. \cite{mccleanTheoryVariationalHybrid2015}), is to use the folded spectrum Hamiltonian \cite{Wang1994}: $\hat{H}^{\prime} = (\hat{H} - \gamma \unit)^2 $, for which the ground state is now the eigenstate of $\hat{H}$ which has an eigenvalue closest to $\gamma$. It was applied by Liu \textit{et al.} \cite{Liu2021_MBL} as a mean to probe many-body localization on a quantum computer. This method however implies squaring the Hamiltonian, which can result in a significant increase in measurements required if the operator is dense, and requires prior knowledge of the eigenspectrum (which is somewhat less of a problem in the case of vibrational spectroscopy than in the case of electronic structure computation \cite{Sawaya2021}). 

The subspace search VQE (SSVQE) \cite{nakanishiSubspacesearchVariationalQuantum2019} leverages the fact that a unitary transformation between states cannot change the orthogonality of the states it is applied to. Therefore by preparing different orthogonal input states and training a VQE ansatz to minimize the energy of all these states at the same time (for instance by modifying the VQE cost function to include the sum of expectation values of the Hamiltonian with respect to each of the states, or by creating a mixed state using ancilla qubits), one can simultaneously learn the ground state and any number of subsequent excited states. It is likely however that this simultaneous optimization of the ansatz becomes increasingly more constrained with the number of excited states desired.

Higgott \textit{et al.} \cite{higgottVariationalQuantumComputation2019} proposed using a deflated Hamiltonian to iteratively compute successive excited states (oftentimes referred to as Variational Quantum Deflation, VQD). The algorithm works by first computing the ground state with VQE. Once discovered, the cost function is modified to add a penalty term, which corresponds to the overlap between the ground state and a new trial wavefunction. This new trial wavefunction is then trained to minimize both the expectation value of the Hamiltonian and maximize the overlap with the lower energy states. This process can be repeated iteratively for any number of excited states. The key challenge of this method is the computation of the overlap term which may require quantum cost that could be significant for a NISQ device (i.e. a large number of SWAP gates), or possibly subject to additional noise (by implementing as a circuit the complex conjugate of the ansatz used to prepare previous excited states, a method also applied in Ref. \cite{Lee2019}), though improvements have been proposed. For instance, Jones \textit{et al.} \cite{Jones2019} propose to compute the overlap term with a low depth SWAP test, and uses variational time evolution \cite{McArdle2019}. Chan \textit{et al.} \cite{Chan2021} extend this excited state method by merging it with ADAPT-VQE \cite{Grimsley2019} (see Sec. \ref{sec:adapt-vqe}). Kottmann \textit{et al.} \cite{Kottmann2021_3} independently also proposed an adaptation of VQD to an adaptive method which benefits from efficiency gained from gradient evaluation process presented in the same work (see Sec. \ref{sec:Optimization}). Wakaura and Suksomo \cite{Wakaura2021} propose an adaptation of the VQE cost function to minimize the norm of the tangent vector to the energy rather than just the energy, dubbed Tangent-Vector VQE (TVVQE). While this can be used for ground state energies, it is also combined with VQD to compute excited states. While the method is shown to provide improved accuracy compared to a UCC based VQE on simple models (Hubbard, $\mathrm{H_2}$, $\mathrm{LiH}$), it is reported to require a run time on average five times longer than VQE \cite{Wakaura2021}. 

The discriminative VQE (DVQE) \cite{Tilly2020} is a further alternative method, and relies on training of a generative adversarial network to enforce orthogonality between the ground state and a trial excited state. Generative Adversarial Networks (GANs) are machine learning tools, which are composed of two neural networks competing against each other: a generator, which is trained to produce a specific data pattern (e.g. an image), and a discriminator, which is trained to distinguish between true instances of this data pattern, and generated instances. When the training is successful, the generator learns to generate data patterns, which are indistinguishable from true ones. This concept was ported to quantum computing with the Quantum GAN \cite{Lloyd2018, Benedetti2019}, a method, which can be used to learn an approximation of an unknown pure state. The DVQE proposed in \cite{Tilly2020} inverts the logic of the QGAN, forcing generator and discriminator to collaborate for the generator to generate a state, which is as easy to distinguish as possible from the ground state: an orthogonal state. In order to ensure that the generated state is the first excited state, one must at the same time minimize the expectation value of the Hamiltonian. Subsequent excited states can be found by repeating the procedure iteratively. The scaling of the depth required for the discriminator remains unknown and could become an impediment for the method.

An alternative approach is the Variance VQE method \cite{zhang2020variance_minimization}, which replaces the usual cost function of VQE by minimizing the variance of a Hamiltonian with respect to a state, rather than its expectation value. The idea behind this method is that the variance of the expectation value of a Hamiltonian must be equal to zero if the state used to perform the measurement is an eigenstate of that Hamiltonian (on the zero-energy variance principle, we direct readers to Ref.~\cite{Bartlett1935, Siringo2005, Umrigar2005, Khemani2016, Pollmann2016, Vicentini2019}). Because all eigenstates have zero-energy variance, a simple approach will not guarantee convergence to a low-energy state. This problem is addressed in Ref.~ \cite{zhang2020variance_minimization} by combining both energy and variance minimization in order to allow for computation of low lying excited states. Zhang \textit{et al.} \cite{zhangAdaptiveVariationalQuantum2021} propose an adaptative variant of this method to computed highly excited states of Hamiltonians. The ansatz is grown by choosing operators from a pool of Pauli operators, akin to the methods listed in Sec. \ref{sec:adaptive_ansatz}.

\paragraph{Dynamical correlation functions}
Equilibrium dynamical correlation functions are the key quantities governing the linear response behavior of quantum systems, encoding the information of the excitation spectrum over all energy scales. These functions can either be represented in the time or frequency (energy) domain, with a key dynamic response function being the single-particle Green's function, describing the charged excitation spectrum of the system. Any method to systematically calculate individual excited state energies (e.g. via a quantum subspace expansion) and the relevant transition amplitudes coupling them to the ground state, can in principle compute these dynamical correlation functions via its spectral representation \cite{Endo2019GF,runggerDynamicalMeanField2020,zhuCalculatingGreenFunction2021,Jamet2021,Jamet2022}. However,  other VQE approaches exist which directly target these correlation functions in either the (imaginary) time or frequency domains \cite{Endo2019GF,sunQuantumComputationFiniteTemperature2021, Wecker2015_Solving}. These include VQE-based variational approaches to directly solve the linear equations for the response of a system at a given frequency \cite{xuVariationalAlgorithmsLinear2021}, which can be cast as a modified cost function, with a similarly parameterized VQE ansatz \cite{Chen2021,caiQuantumComputationMolecular2020, Tong2021}. While these approaches can describe the correlation functions to high accuracy over the whole energy range without restricting to a low-energy subspace, their challenge arises chiefly from the substantially more difficult optimization problem for the ansatz, originating from the larger condition number of the cost function, as well as the necessity for Hadamard tests to compute the transition amplitudes between the excited (or response) and ground states (for more details, see Appendix.~\ref{sec:hadamard-test}).

%A number of approaches have been recently proposed for computing equilibrium dynamical correlation functions (and more specifically the equilibrium Green's function) on a quantum computer. Here we gave a short overview of the proposed methods that are NISQ friendly. They can be broadly categorized into whether they compute the function in the time or frequency domain.
%For computing the correlation function in the time domain, noise-resilient methods have been proposed based on variational quantum simulation algorithms \citet{Endo2019GF} or on quantum imaginary time evolution algorithms \cite{sunQuantumComputationFiniteTemperature2021} to compute the Green's function in the time domain. Both algorithms are variational quantum algorithms that rely on a good enough ansatz to capture the quantum correlation and are closely related to the VQE algorithm discussed here. Similarly, in the frequency domain, some use VQE to calculate excited states to construct Green's functions via its spectral representation\cite{runggerDynamicalMeanField2020, Endo2019GF, zhuCalculatingGreenFunction2021} or in a Krylov subspace\cite{Jamet2021}, while others\cite{Chen2021,caiQuantumComputationMolecular2020} use VQE-based variational linear algebra algorithms\cite{xuVariationalAlgorithmsLinear2021} to compute Green's function exactly for each frequency point.

\subsection{VQE as a solver of correlated subspaces in multiscale methods} \label{sec:multiscale}

The VQE has been applied as a sub-routine to resolve the low-energy electronic structure in a number of existing approaches, thereby adapting many hybrid methods of conventional quantum chemistry methods to exploit quantum computing. These include a number of `quantum embedding' methods, where the full space of the problem is partitioned, with each solved at a differing level of theory. In these, it is generally the strongly correlated low-energy partition of orbitals that are amenable to use within a VQE solver which are  then, in various ways, coupled back to the rest of the system (potentially self-consistently) at a lower level of theory on a classical device. These multi-resolution methods can substantially extend the scope and applicability of the VQE, under additional constraints arising from this choice of partitioning and coupling of the spaces. We have provided below three examples of embedding methods adapted in this way:

\paragraph{Complete active space approaches:} 
The simplest and most widespread approach in quantum chemistry for isolating and treating a correlated set of low-energy degrees of freedom at a higher level of theory are the Complete Active Space (CAS) approaches. In these, a subset of high-energy occupied and low-energy unoccupied Hartree--Fock orbitals are considered to span the dominant strongly correlated quantum fluctuations, and treated with an accurate correlated treatment within this subspace (often full configuration interaction, see Sec.~\ref{sec:full_configuration_interaction}). This subspace Hamiltonian includes the presence of a Coulomb and exchange mean-field potential from the remaining electrons outside this space. In this way, the active space electrons are fully correlated within that manifold, leading to the Complete Active Space Configuration Interaction (CAS-CI) approach \cite{Jensen2017, Levine2021}. Furthermore, the CAS-CI can be variationally optimized, by updating the choice of molecular orbitals defining the low-energy CAS space via single-particle unitary rotations among the entire set of orbitals in the system. 
This method is generally referred to as Complete Active Space Self-Consistent Field (CASSCF) \cite{Roos1980, Sun2017}, or the related Multi-Configurational Self-Consistent Field (MCSCF) where the active space is not solved at the level of full configuration interaction. The CASSCF wavefunction can therefore be written as follows:
\begin{align}
\label{eq:QCAS_WF_1}
    |\Psi_{\mathrm{CASSCF}} \rangle = \ket{\textbf{R}, \textbf{c}} = e^{-\textbf{R}} \sum_{\mu} c_{\mu} \ket{\mu},
\end{align}
where $\textbf{R}$ parameterizes the single-particle anti-unitary operator defining the rotation of the active space, $\ket{\mu}$ the complete set of Slater determinants in the active space, and $\textbf{c}$ defines the coefficients of the configurations indexed by $\mu$. In implementation on a quantum device, the rotation operator defining the active space, $\textbf{R}$, can be optimized on a classical device, while the parameterized description of the active space wavefunction can be sought via the VQE. These approaches constitute the bedrock for simulation of molecular systems with strong correlation, in particular in systems with competing spin states, excited states, systems at bond-breaking geometries, and inorganic chemistry \cite{RetaManeru2014,Li2015,olsen11}. These CAS-based approaches were initially proposed in combination with VQE as a solver for the active space in Ref.~\cite{reiherElucidatingReactionMechanisms2017} and were subsequently successfully demonstrated practically on quantum computers in Refs.~\cite{Takeshita2020, Yalouz2021, Tilly2021}, including self-consistent optimization of the active space. 

It should be noted that in order to achieve this optimization of the active space, the two-body reduced density matrix of the active space is required, which can have ramifications on the number of measurements required by the VQE \cite{Tilly2021}. However, in strongly correlated quantum chemistry, it is generally also important to include a description of the correlation within the orbitals external to the active space, generally via low-order perturbation theory, resulting in methods such as complete active space second-order perturbation theory (CASPT2) \cite{Abe2008}. These however require computation of the 3-body reduced density matrix (and potentially higher) in order to couple the active space correlations to this perturbative treatment and are therefore considered a daunting proposition for VQE. There is also a wider range of extensions to the CASCI approach, including extensions to embedding with density functional theory (DFT) description of the environment, which has also been explored by Rossmannek et al. \cite{Rossmannek2021} within a VQE description of a correlated active space. Shade \textit{et al.} \cite{Schade2022} also extend these ideas to the reduced density matrix function theory (RDMFT) and demonstrate an implementation of their method to a Hubbard-like system on a quantum device.

\paragraph{Density matrix embedding theory (DMET):} Similar to the active space methods mentioned above, DMET~\cite{Knizia2013, Wouters2016} aims at embedding an accurately correlated subspace in a mean-field environment. In contrast to CAS-CI, this `active space' is chosen through locality criteria, starting from a local fragment space and augmenting it with the minimal number of additional orbitals (denoted the bath space) to ensure that the active space recovers the Hartree--Fock description, and explicitly captures quantum entanglement between the fragment and its environment. In this way, the DMET approach can be considered as having a similar ambition to dynamical mean-field theory \cite{Georges1996}, but cast as a static wavefunction theory (see below). In order to optimize the mean-field state of the system, the one-body reduced density matrix is matched between the individual fragment spaces between the correlated and mean-field descriptions.

Integration of DMET with a VQE for the correlated subspace solver has been the subject of several publications \cite{Rubin2016, Yamazaki2018, Ma2020, Mineh2021, Li2021}, and has been implemented on quantum computers with proof of principles for relevant applications such as protein-ligand interactions for drug design \cite{Kirsopp2021} (with an alternative method based on perturbation theory proposed in Ref.~\cite{Malone2021}). Energy weighted DMET (EwDMET) which builds on DMET to improve its description of dynamical fluctuations for small fragment sizes (thereby moving systematically towards a DMFT description described below) \cite{Fertitta2018, Fertitta2019, Sriluckshmy2021} was also tested and implemented on a quantum device \cite{Tilly2021}, allowing quantum phase transitions to be captured which were out of the scope of DMET. A wide range of possible alternative formulations exist for embedding correlated subspaces in (static) mean-field environments - especially when that subspace is only weakly coupled to the environment, and the explicit entanglement between the subspace and the environment can be neglected.

\paragraph{Dynamical Mean-Field Theory (DMFT):} DMFT again relies on a similar embedding of a (local) correlated subspace in a mean-field environment. However, this environment allows for local quantum fluctuations in its description, thereby including the effects of correlation in the local propagation of particles through the environment. This effect is captured by a local self-energy, which is the self-consistent quantum object in DMFT \cite{Georges1996}. This necessitates a formalism built around the single-particle Green's function (a specific dynamical correlation function), which is the object which must be sampled within DMFT on a quantum device. At the heart of DMFT is a mapping from the system of interest to an impurity model, which describes a local correlated fragment coupled to a wider non-interacting set of degrees of freedom, denoted the `bath'. This impurity model can be represented in a Hamiltonian formulation, from which the single-particle Green's function must be sampled, with various approaches to solve for this Green's function known as `impurity solvers'. The techniques presented earlier in this section for can be used to sample this Green's function in either a time or frequency domain at each iteration in the self-consistent loop. 
The use of quantum computers as an impurity solver was proposed initially in Ref. \cite{Bauer2016} in the time domain, but frequency domain solvers have often been more amenable to the low-depth NISQ era. These were explored in the context of DMFT impurity solvers in physical realizations of correlated material systems via VQE-type parameterized algorithms in Refs.~\cite{Endo2019GF,runggerDynamicalMeanField2020, Kreula2016,Jamet2021}. An alternative method to compute the Green's function over the whole energy range is based on the quantum subspace expansion \cite{Jamet2022}.

Overall, embedding methods using the VQE as a high accuracy and scalable solver to describe the correlations within a subspace self-consistently coupled to a wider environment are a promising avenue to extend the applicability of quantum computation towards practical applications. In general, they allow for recovery of significant parts of the electron correlation energy, while avoiding treatment of the full system, thereby reducing qubits number in exchange for additional classical resources in defining the embedding, as well as a self-consistent loop. It is worth noting that the possibilities for embedding the VQE and more general quantum algorithms within wider multi-method and multi-resolution hybrid schemes extends far beyond just the quantum embedding methodologies presented above, and are likely to be of central importance in the utility of quantum algorithms in molecular modeling in all contexts in the future.