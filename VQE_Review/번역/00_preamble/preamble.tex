% --------------------------------------------------------------------
% # README
%
% Here we puts the common elements that should be put into the preamble part
% of a latex document.
% Various latex files for compiling a single section should % --------------------------------------------------------------------
% # README
%
% Here we puts the common elements that should be put into the preamble part
% of a latex document.
% Various latex files for compiling a single section should % --------------------------------------------------------------------
% # README
%
% Here we puts the common elements that should be put into the preamble part
% of a latex document.
% Various latex files for compiling a single section should % --------------------------------------------------------------------
% # README
%
% Here we puts the common elements that should be put into the preamble part
% of a latex document.
% Various latex files for compiling a single section should \input{preamble.tex}
% in their preamble part.
% --------------------------------------------------------------------
% usepackage part
% --------------------------------------------------------------------
\usepackage[utf8]{inputenc}
\usepackage[T1]{fontenc}
\usepackage[english]{babel}
\usepackage{graphicx}% Include figure files
\usepackage{dcolumn}% Align table columns on decimal point
% NOTE: physics must be loaded before braket. Since physics redfines
% \braket in a different manner.
\usepackage{physics}
\usepackage{braket}
\usepackage{amsmath}
% several options for the identify operator:
% 1. \usepackage{bbm} % for the identity operator \mathbb{1}
% 2. \usepackage{bm}% bold math
% 3. use dsfont (the current choice)
\usepackage{dsfont} % for \mathds{1}, i.e. \unit
\newcommand{\unit}{\mathds{1}}  % defines \unit to be the identify operator.
\usepackage{appendix}
\usepackage[dvipsnames]{xcolor}
\usepackage{pgf}
\usepackage{hyperref}
\hypersetup{ % makes hyperref's links friendly to read
    colorlinks=true,
    linkcolor=blue,
    filecolor=blue,      
    urlcolor=blue,
}
\usepackage{blochsphere}
\usepackage[colorinlistoftodos]{todonotes}
\usepackage{qcircuit} % For drawing quantum circuits
\usepackage{blkarray}
\usepackage{booktabs}
\usepackage{tabularx} % For mimicking ruledtabular in revtex
\usepackage{multirow} % creating multi-row items inside tables.
\usepackage{longtable} % For creating multipage tables
\usepackage{mathtools}
\usepackage[ruled]{algorithm2e} % For algorithm environment
% \usepackage[intoc]{nomencl} % for list of symbols
% \makenomenclature % for list of symbols
\usepackage[numbers, sort&compress]{natbib} % for citation using numbered style
% For subfigures.
% Note that subfig requires the option caption=false for compatibility with revtex (though it's not needed now).
\usepackage{subfig}
% --------------------------------------------------------------------
% Custom commands regions:
% --------------------------------------------------------------------
\newcommand{\lw}[1]{\textcolor{green}{#1}}
\newcommand{\hx}[1]{\textcolor{cyan}{(\small{hx:}#1)}}
\newcommand{\eg}[1]{\textcolor{red}{ed:#1}}
\newcommand{\sx}[1]{\textcolor{teal}{sx:#1}}
\newcommand{\ghb}[1]{\textcolor{red}{ghb:#1}}
% For << and >> in Probablisitc Error Cancellation in error mitigation.
\newcommand{\rang}{\rangle\!\rangle}
\newcommand{\lang}{\langle\!\langle}
% For argmax and argmin and etc.
\DeclareMathOperator*{\argmax}{\arg\!\max}
\DeclareMathOperator*{\argmin}{\arg\!\min}
\DeclareMathOperator{\arctantwo}{arctan2}
% For unknown purposes: ???
\setlength{\skiptext}{10pt}
\setlength{\skiprule}{5pt}
% Defines a path pointing to project's root, which can be re-defined
% if needed. This helps the latex code in Per Section Compilation folder.
\newcommand*{\ProjectRoot}{.}
% Defines citet as Ref~\cite, since we are using numbered style citation.
\renewcommand{\citet}[1]{Ref.~\cite{#1}}
% This increases the spacing between rows in a table by a relative factor: 
\renewcommand{\arraystretch}{1.2}
% a symbol for Hadamard, QFT, equivalent to \operatorname{xxx}
\DeclareMathOperator{\Had}{Had}
\DeclareMathOperator{\QFT}{QFT}
% \nEES is the symbol for the order of virtual distillation in the exponential error suppression section.
\newcommand{\nEES}{K}
% in their preamble part.
% --------------------------------------------------------------------
% usepackage part
% --------------------------------------------------------------------
\usepackage[utf8]{inputenc}
\usepackage[T1]{fontenc}
\usepackage[english]{babel}
\usepackage{graphicx}% Include figure files
\usepackage{dcolumn}% Align table columns on decimal point
% NOTE: physics must be loaded before braket. Since physics redfines
% \braket in a different manner.
\usepackage{physics}
\usepackage{braket}
\usepackage{amsmath}
% several options for the identify operator:
% 1. \usepackage{bbm} % for the identity operator \mathbb{1}
% 2. \usepackage{bm}% bold math
% 3. use dsfont (the current choice)
\usepackage{dsfont} % for \mathds{1}, i.e. \unit
\newcommand{\unit}{\mathds{1}}  % defines \unit to be the identify operator.
\usepackage{appendix}
\usepackage[dvipsnames]{xcolor}
\usepackage{pgf}
\usepackage{hyperref}
\hypersetup{ % makes hyperref's links friendly to read
    colorlinks=true,
    linkcolor=blue,
    filecolor=blue,      
    urlcolor=blue,
}
\usepackage{blochsphere}
\usepackage[colorinlistoftodos]{todonotes}
\usepackage{qcircuit} % For drawing quantum circuits
\usepackage{blkarray}
\usepackage{booktabs}
\usepackage{tabularx} % For mimicking ruledtabular in revtex
\usepackage{multirow} % creating multi-row items inside tables.
\usepackage{longtable} % For creating multipage tables
\usepackage{mathtools}
\usepackage[ruled]{algorithm2e} % For algorithm environment
% \usepackage[intoc]{nomencl} % for list of symbols
% \makenomenclature % for list of symbols
\usepackage[numbers, sort&compress]{natbib} % for citation using numbered style
% For subfigures.
% Note that subfig requires the option caption=false for compatibility with revtex (though it's not needed now).
\usepackage{subfig}
% --------------------------------------------------------------------
% Custom commands regions:
% --------------------------------------------------------------------
\newcommand{\lw}[1]{\textcolor{green}{#1}}
\newcommand{\hx}[1]{\textcolor{cyan}{(\small{hx:}#1)}}
\newcommand{\eg}[1]{\textcolor{red}{ed:#1}}
\newcommand{\sx}[1]{\textcolor{teal}{sx:#1}}
\newcommand{\ghb}[1]{\textcolor{red}{ghb:#1}}
% For << and >> in Probablisitc Error Cancellation in error mitigation.
\newcommand{\rang}{\rangle\!\rangle}
\newcommand{\lang}{\langle\!\langle}
% For argmax and argmin and etc.
\DeclareMathOperator*{\argmax}{\arg\!\max}
\DeclareMathOperator*{\argmin}{\arg\!\min}
\DeclareMathOperator{\arctantwo}{arctan2}
% For unknown purposes: ???
\setlength{\skiptext}{10pt}
\setlength{\skiprule}{5pt}
% Defines a path pointing to project's root, which can be re-defined
% if needed. This helps the latex code in Per Section Compilation folder.
\newcommand*{\ProjectRoot}{.}
% Defines citet as Ref~\cite, since we are using numbered style citation.
\renewcommand{\citet}[1]{Ref.~\cite{#1}}
% This increases the spacing between rows in a table by a relative factor: 
\renewcommand{\arraystretch}{1.2}
% a symbol for Hadamard, QFT, equivalent to \operatorname{xxx}
\DeclareMathOperator{\Had}{Had}
\DeclareMathOperator{\QFT}{QFT}
% \nEES is the symbol for the order of virtual distillation in the exponential error suppression section.
\newcommand{\nEES}{K}
% in their preamble part.
% --------------------------------------------------------------------
% usepackage part
% --------------------------------------------------------------------
\usepackage[utf8]{inputenc}
\usepackage[T1]{fontenc}
\usepackage[english]{babel}
\usepackage{graphicx}% Include figure files
\usepackage{dcolumn}% Align table columns on decimal point
% NOTE: physics must be loaded before braket. Since physics redfines
% \braket in a different manner.
\usepackage{physics}
\usepackage{braket}
\usepackage{amsmath}
% several options for the identify operator:
% 1. \usepackage{bbm} % for the identity operator \mathbb{1}
% 2. \usepackage{bm}% bold math
% 3. use dsfont (the current choice)
\usepackage{dsfont} % for \mathds{1}, i.e. \unit
\newcommand{\unit}{\mathds{1}}  % defines \unit to be the identify operator.
\usepackage{appendix}
\usepackage[dvipsnames]{xcolor}
\usepackage{pgf}
\usepackage{hyperref}
\hypersetup{ % makes hyperref's links friendly to read
    colorlinks=true,
    linkcolor=blue,
    filecolor=blue,      
    urlcolor=blue,
}
\usepackage{blochsphere}
\usepackage[colorinlistoftodos]{todonotes}
\usepackage{qcircuit} % For drawing quantum circuits
\usepackage{blkarray}
\usepackage{booktabs}
\usepackage{tabularx} % For mimicking ruledtabular in revtex
\usepackage{multirow} % creating multi-row items inside tables.
\usepackage{longtable} % For creating multipage tables
\usepackage{mathtools}
\usepackage[ruled]{algorithm2e} % For algorithm environment
% \usepackage[intoc]{nomencl} % for list of symbols
% \makenomenclature % for list of symbols
\usepackage[numbers, sort&compress]{natbib} % for citation using numbered style
% For subfigures.
% Note that subfig requires the option caption=false for compatibility with revtex (though it's not needed now).
\usepackage{subfig}
% --------------------------------------------------------------------
% Custom commands regions:
% --------------------------------------------------------------------
\newcommand{\lw}[1]{\textcolor{green}{#1}}
\newcommand{\hx}[1]{\textcolor{cyan}{(\small{hx:}#1)}}
\newcommand{\eg}[1]{\textcolor{red}{ed:#1}}
\newcommand{\sx}[1]{\textcolor{teal}{sx:#1}}
\newcommand{\ghb}[1]{\textcolor{red}{ghb:#1}}
% For << and >> in Probablisitc Error Cancellation in error mitigation.
\newcommand{\rang}{\rangle\!\rangle}
\newcommand{\lang}{\langle\!\langle}
% For argmax and argmin and etc.
\DeclareMathOperator*{\argmax}{\arg\!\max}
\DeclareMathOperator*{\argmin}{\arg\!\min}
\DeclareMathOperator{\arctantwo}{arctan2}
% For unknown purposes: ???
\setlength{\skiptext}{10pt}
\setlength{\skiprule}{5pt}
% Defines a path pointing to project's root, which can be re-defined
% if needed. This helps the latex code in Per Section Compilation folder.
\newcommand*{\ProjectRoot}{.}
% Defines citet as Ref~\cite, since we are using numbered style citation.
\renewcommand{\citet}[1]{Ref.~\cite{#1}}
% This increases the spacing between rows in a table by a relative factor: 
\renewcommand{\arraystretch}{1.2}
% a symbol for Hadamard, QFT, equivalent to \operatorname{xxx}
\DeclareMathOperator{\Had}{Had}
\DeclareMathOperator{\QFT}{QFT}
% \nEES is the symbol for the order of virtual distillation in the exponential error suppression section.
\newcommand{\nEES}{K}
% in their preamble part.
% --------------------------------------------------------------------
% usepackage part
% --------------------------------------------------------------------
\usepackage[utf8]{inputenc}
\usepackage[T1]{fontenc}
\usepackage[english]{babel}
\usepackage{graphicx}% Include figure files
\usepackage{dcolumn}% Align table columns on decimal point
% NOTE: physics must be loaded before braket. Since physics redfines
% \braket in a different manner.
\usepackage{physics}
\usepackage{braket}
\usepackage{amsmath}
% several options for the identify operator:
% 1. \usepackage{bbm} % for the identity operator \mathbb{1}
% 2. \usepackage{bm}% bold math
% 3. use dsfont (the current choice)
\usepackage{dsfont} % for \mathds{1}, i.e. \unit
\newcommand{\unit}{\mathds{1}}  % defines \unit to be the identify operator.
\usepackage{appendix}
\usepackage[dvipsnames]{xcolor}
\usepackage{pgf}
\usepackage{hyperref}
\hypersetup{ % makes hyperref's links friendly to read
    colorlinks=true,
    linkcolor=blue,
    filecolor=blue,      
    urlcolor=blue,
}
\usepackage{blochsphere}
\usepackage[colorinlistoftodos]{todonotes}
\usepackage{qcircuit} % For drawing quantum circuits
\usepackage{blkarray}
\usepackage{booktabs}
\usepackage{tabularx} % For mimicking ruledtabular in revtex
\usepackage{multirow} % creating multi-row items inside tables.
\usepackage{longtable} % For creating multipage tables
\usepackage{mathtools}
\usepackage[ruled]{algorithm2e} % For algorithm environment
% \usepackage[intoc]{nomencl} % for list of symbols
% \makenomenclature % for list of symbols
\usepackage[numbers, sort&compress]{natbib} % for citation using numbered style
% For subfigures.
% Note that subfig requires the option caption=false for compatibility with revtex (though it's not needed now).
\usepackage{subfig}
% --------------------------------------------------------------------
% Custom commands regions:
% --------------------------------------------------------------------
\newcommand{\lw}[1]{\textcolor{green}{#1}}
\newcommand{\hx}[1]{\textcolor{cyan}{(\small{hx:}#1)}}
\newcommand{\eg}[1]{\textcolor{red}{ed:#1}}
\newcommand{\sx}[1]{\textcolor{teal}{sx:#1}}
\newcommand{\ghb}[1]{\textcolor{red}{ghb:#1}}
% For << and >> in Probablisitc Error Cancellation in error mitigation.
\newcommand{\rang}{\rangle\!\rangle}
\newcommand{\lang}{\langle\!\langle}
% For argmax and argmin and etc.
\DeclareMathOperator*{\argmax}{\arg\!\max}
\DeclareMathOperator*{\argmin}{\arg\!\min}
\DeclareMathOperator{\arctantwo}{arctan2}
% For unknown purposes: ???
\setlength{\skiptext}{10pt}
\setlength{\skiprule}{5pt}
% Defines a path pointing to project's root, which can be re-defined
% if needed. This helps the latex code in Per Section Compilation folder.
\newcommand*{\ProjectRoot}{.}
% Defines citet as Ref~\cite, since we are using numbered style citation.
\renewcommand{\citet}[1]{Ref.~\cite{#1}}
% This increases the spacing between rows in a table by a relative factor: 
\renewcommand{\arraystretch}{1.2}
% a symbol for Hadamard, QFT, equivalent to \operatorname{xxx}
\DeclareMathOperator{\Had}{Had}
\DeclareMathOperator{\QFT}{QFT}
% \nEES is the symbol for the order of virtual distillation in the exponential error suppression section.
\newcommand{\nEES}{K}